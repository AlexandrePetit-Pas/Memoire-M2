\documentclass[memoire.tex]{subfiles}

\chapter*{Introduction}
\addcontentsline{toc}{chapter}{Introduction}

\subsection*{Motivations}
J'ai choisi ce sujet suite à trois événements. Durant l'un de mes stages, j'ai pu travailler avec un doctorant en informatique qui faisait des recherches sur de l'apprentissage. C'est aussi durant ce stage que j'ai approché pour la première fois l'informatique décisionnelle. Ce sujet m'a rapidement intéressé étant donné la grandeur et la popularité de celui-ci. Cependant, je ne voulais pas l'aborder sous le thème de la recommandation (qui est populaire mais a été traité de nombreuses fois), c'est pourquoi je me suis intéressé en profondeur à ce qui se faisait en terme d'arbre de décision.\\

C'est en découvrant l'existence d'algorithmes \textit{lazy} et des \textit{Lazy Decision Trees} que le sujet a été choisi. Les algorithmes paresseux, ou \textit{lazy}, sont beaucoup utilisés dans le domaine du Web pour leur économie en ressource. Les ORM utilisent beaucoup cette technique qui leur permet de minimiser les requêtes vers la base de données. C'est pourquoi j'ai trouvé intéressant le fait d'apporter ce concept dans l'apprentissage supervisé.\\

Par ailleurs, durant mon stage de troisième année de licence, j'ai eu l'occasion de surveiller des partiels de première année (L1). Leurs épreuves étaient sous forme de QCM avec certaines réponses libres. Toutes les questions fermées étaient corrigées automatiquement alors que les réponses libres ne l'étaient pas. Un autre problème qui m'a marqué était l'écriture du numéro étudiant, composé de chiffres. Celui-ci n'était pas récupéré automatiquement car l'outil ne pouvait pas reconnaître les chiffres écrits à la main.\\

Ce sont ces trois idées qui m'ont permis d'écrire ce mémoire, afin de comprendre le domaine de l'apprentissage supervisé en répondant à un besoin d'enseignants universitaires.

\subsection*{Problématique}
La reconnaissance d'images est un domaine populaire qui est devenu mature avec le temps. Des problématiques de performance continuent d'exister de différentes manières : l'image reconnue est-elle la bonne et l'algorithme est-il suffisamment performant pour supporter la charge ? Dans certains contextes, l'erreur peut coûter très cher et dans d'autres, il n'est pas possible de reconnaître efficacement l'image. Nous allons nous intéresser à la reconnaissance de chiffres provenant du MNIST. Cependant, il est possible par extension de l'appliquer sur d'autres jeux de données. La problématique est de savoir si les solutions recherchées sont suffisamment efficaces pour être appliquées dans un cas concret de reconnaissance de numéros étudiants à des fins d'évaluation.

\subsection*{Objectifs}
L'objectif de ce mémoire est de comprendre comment les techniques de classement fonctionnent et de les appliquer sur un ensemble de données connues. Les résultats permettront alors de connaître si un algorithme est plus performant qu'un autre et d'apporter une réflexion autour de l'amélioration de ceux-ci. Il est disponible à l'adresse suivante~\url{https://github.com/AlexandrePetit-Pas/Memoire-M2}