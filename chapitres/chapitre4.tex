\documentclass[memoire.tex]{subfiles}

\chapter{Apprentissage du MNIST}
\section{Introduction}

\section{Définition du MNIST}
MINST (\textit{Modified National Institute of Standards and Technology}) est une base de données de chiffres, écrits à la main, produite par~\citeauthor{mnist}~\cite{mnist}. Elle s'inspire d'une autre base de données, créée par le NIST \textit{National Institute of Standards and Technology}, qui possédait des caractères alphanumériques écrits à la main. La base MNIST est composée de 60 000 images d'apprentissage et 10 000 images de test, chacune en noir et blanc et d'une taille de 28x28 pixel. Une technique d’anticrénelage a été appliqué sur les images afin de leur donner un niveau de gris~\cite{mnist}. Une image donne alors un vecteur de dimension 784 dont chaque valeur est comprise entre 0 (blanc) et 255 (noir).\\

La base MNIST est une base qui sert de standard dans les domaines de l'apprentissage de part sa disponibilité, sa taille et le fait que les données soient des caractères écrits à la main~\cite{mnist2}. La figure 4.1 montre un exemple d'une observation du jeu de données, qui est un chiffre 5.

\begin{figure}[!h]
	\centering
	\includegraphics[scale=1]{img/5.png}
	\caption{Exemple d'un chiffre de la base MNIST}
\end{figure}

\section{Méthode de recherche}
\section{Algorithme de classement}
\section{Algorithme Lazy}
\section{Comparaison des résultats}