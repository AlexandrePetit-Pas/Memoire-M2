\documentclass[12pt, twoside, openright]{report}

%----------------------------------------------------------------------------------------
%	PACKAGES
%----------------------------------------------------------------------------------------

\usepackage{emptypage}
\usepackage{geometry}  	
\usepackage[utf8x]{inputenc}
\usepackage[french]{babel}
\usepackage[T1]{fontenc}
\usepackage{array}
\usepackage{amsmath}
\usepackage{amsfonts}
\usepackage{amssymb}
\usepackage{graphicx}
\usepackage{subfiles}
\usepackage{fullpage}
\usepackage{fancyhdr}
\usepackage{shorttoc}
\usepackage{fontspec}
\usepackage{makecell}
\usepackage{xltxtra}
\usepackage{xcolor}
\usepackage{pgfplots}
\usepackage[]{algorithm2e}
\usepackage{sectsty}
\usepackage[Lenny]{fncychap}
\usepackage[backend=bibtex,sorting=none]{biblatex}

%----------------------------------------------------------------------------------------
%	BIBLIOGRAPHIE
%----------------------------------------------------------------------------------------

\addbibresource{bibliographie/bibliographie.bib}

%----------------------------------------------------------------------------------------
%	STYLES
%----------------------------------------------------------------------------------------

\ChNumVar{\fontsize{60}{62}\usefont{OT1}{ptm}{m}{n}\selectfont\textcolor{red}}

\setmainfont[Mapping=tex-text]{Lato}

\geometry{
    paper=a4paper,
    inner=3cm,
    outer=2.5cm,
    top=2.5cm,
    bottom=3.5cm
}

\pagestyle{fancy}
\usepackage{etoolbox}
\patchcmd{\chapter}{\thispagestyle{plain}}{\thispagestyle{fancy}}{}{}
\renewcommand\headrulewidth{0pt}
\fancyhead[L]{}
\fancyhead[C]{}
\fancyhead[R]{}

\makeatletter
\patchcmd{\@makechapterhead}{\vspace*{50\p@}}{\vspace*{-35\p@}}{}{}
\patchcmd{\@makeschapterhead}{\vspace*{50\p@}}{\vspace*{-35\p@}}{}{}
\patchcmd{\DOTI}{\vskip 80\p@}{\vskip 40\p@}{}{}
\patchcmd{\DOTIS}{\vskip 40\p@}{\vskip 0\p@}{}{}
\makeatother

\renewcommand\thesection{\color{red}\thechapter.\arabic{section}}

\newenvironment{acknowledgements} {\renewcommand\abstractname{Remerciements}\begin{abstract}} {\end{abstract}}

%----------------------------------------------------------------------------------------
%	INFORMATIONS
%----------------------------------------------------------------------------------------

\author{Alexandre Petit-Pas}
\title{Mémoire de fin d'études}

\begin{document}

%\subfile{cover/page_garde}
%----------------------------------------------------------------------------------------
%	TITLE PAGE
%----------------------------------------------------------------------------------------

\setlength{\parindent}{0cm}
\setlength{\parskip}{1ex plus 0.5ex minus 0.2ex}
\newcommand{\hsp}{\hspace{20pt}}
\newcommand{\HRule}{\rule{\linewidth}{0.5mm}}

\begin{titlepage}
  \begin{sffamily}
  \begin{center}

    \textsc{\LARGE MASTER MIAGE 2ème année \linebreak Université Paris Nanterre}\\[2cm]

    \textsc{\Large Mémoire de fin d'études}\\[1.5cm]

    \HRule \\[0.4cm]
    { \huge \bfseries Étude comparative d'algorithmes pour le classement de chiffres manuscrits\\[0.4cm] }

    \HRule \\[2cm]
    \vfill
    \begin{center}
    \includegraphics[scale=0.40]{img/logo_nanterre.jpg}
    \end{center}
    \hspace{2cm}
    
    \vfill
  \begin{minipage}{0.3\textwidth}
      \begin{flushleft} \large
        \emph{Auteur :}\\ \textsc{Alexandre PETIT-PAS}\\
      \end{flushleft}
    \end{minipage}
    \begin{minipage}{0.4\textwidth}
      \begin{flushright} \large
        \emph{Tuteur :}\\ \textsc{PR. Marie-Pierre GERVAIS}\\
      \end{flushright}
    \end{minipage}
    \vfill
  \end{center}
  \end{sffamily}
\end{titlepage}

\leavevmode\thispagestyle{empty}\newpage

%----------------------------------------------------------------------------------------
%	RESUME
%----------------------------------------------------------------------------------------

\begin{abstract}
Dans ce mémoire, nous présentons le classement, une méthode précise de l'apprentissage supervisé. Corriger des copies d'examen est une activité chronophage qui pourrait être automatisée. C'est pourquoi il est possible d'implémenter des outils permettant de corriger des questions, exercices ou QCM. Ce mémoire porte sur la partie reconnaissance de chiffres manuscrits permettant ainsi la reconnaissance des numéros étudiants. Trois algorithmes seront étudiés : CART, C4.5 et K-NN. Après analyse et implémentation, nous concluons que CART et K-NN sont deux solutions avec leurs avantages et leurs défauts. Finalement, K-NN sera retenu comme algorithme adéquat pour reconnaître les numéros étudiants efficacement dans un contexte de correction.
\end{abstract}

\leavevmode\thispagestyle{empty}\newpage

%----------------------------------------------------------------------------------------
%	Remerciement
%----------------------------------------------------------------------------------------

\begin{acknowledgements}
Je tiens à remercier Mme Marie-Pierre Gervais et M. Emmanuel Hyon pour l'aide et les conseils précieux qu'ils m'ont apportés durant la rédaction de ce mémoire. 

Un grand merci à toute ma famille qui m'a soutenu jusqu'à la dernière ligne et qui m'a aidé à la relecture.

Enfin je tiens à remercier toutes les personnes qui m'ont aidé d'une quelconque manière à l'écriture de ce mémoire.
\end{acknowledgements}

\leavevmode\thispagestyle{empty}\newpage

%----------------------------------------------------------------------------------------
%	SOMMAIRE
%----------------------------------------------------------------------------------------

\shorttoc{Sommaire}{1}

%----------------------------------------------------------------------------------------
%	INTRODUCTION
%----------------------------------------------------------------------------------------

\subfile{chapitres/introduction}

%----------------------------------------------------------------------------------------
%	CHAPITRES
%----------------------------------------------------------------------------------------
\subfile{chapitres/chapitre1}

\subfile{chapitres/chapitre2}

\subfile{chapitres/chapitre3}


%----------------------------------------------------------------------------------------
%	CONCLUSION
%----------------------------------------------------------------------------------------
\subfile{chapitres/conclusion}

%----------------------------------------------------------------------------------------
%	BIBLIOGRAPHIE
%----------------------------------------------------------------------------------------

\printbibliography[heading=bibintoc]
\newpage

%----------------------------------------------------------------------------------------
%	TABLE DES MATIERES
%----------------------------------------------------------------------------------------

\tableofcontents

\listoffigures
\listoftables

\end{document}