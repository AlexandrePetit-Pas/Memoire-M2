\documentclass[12pt, twoside, openright]{report}

%----------------------------------------------------------------------------------------
%	PACKAGES
%----------------------------------------------------------------------------------------

\usepackage{emptypage}
\usepackage{geometry}  	
\usepackage[utf8x]{inputenc}
\usepackage[french]{babel}
\usepackage[T1]{fontenc}
\usepackage{array}
\usepackage{amsmath}
\usepackage{amsfonts}
\usepackage{amssymb}
\usepackage{graphicx}
\usepackage{subfiles}
\usepackage{fullpage}
\usepackage{fancyhdr}
\usepackage{shorttoc}
\usepackage{fontspec}
\usepackage{makecell}
\usepackage{xltxtra}
\usepackage{xcolor}
\usepackage{pgfplots}
\usepackage[]{algorithm2e}
\usepackage{sectsty}
\usepackage[Lenny]{fncychap}
\usepackage[backend=bibtex,sorting=none]{biblatex}

%----------------------------------------------------------------------------------------
%	BIBLIOGRAPHIE
%----------------------------------------------------------------------------------------

\addbibresource{bibliographie/bibliographie.bib}

%----------------------------------------------------------------------------------------
%	STYLES
%----------------------------------------------------------------------------------------

\ChNumVar{\fontsize{60}{62}\usefont{OT1}{ptm}{m}{n}\selectfont\textcolor{red}}

\setmainfont[Mapping=tex-text]{Lato}

\geometry{
    paper=a4paper,
    inner=3cm,
    outer=2.5cm,
    top=2.5cm,
    bottom=3.5cm
}

\pagestyle{fancy}
\usepackage{etoolbox}
\patchcmd{\chapter}{\thispagestyle{plain}}{\thispagestyle{fancy}}{}{}
\renewcommand\headrulewidth{0pt}
\fancyhead[L]{}
\fancyhead[C]{}
\fancyhead[R]{}

\makeatletter
\patchcmd{\@makechapterhead}{\vspace*{50\p@}}{\vspace*{-35\p@}}{}{}
\patchcmd{\@makeschapterhead}{\vspace*{50\p@}}{\vspace*{-35\p@}}{}{}
\patchcmd{\DOTI}{\vskip 80\p@}{\vskip 40\p@}{}{}
\patchcmd{\DOTIS}{\vskip 40\p@}{\vskip 0\p@}{}{}
\makeatother

\renewcommand\thesection{\color{red}\thechapter.\arabic{section}}

\newenvironment{acknowledgements} {\renewcommand\abstractname{Remerciements}\begin{abstract}} {\end{abstract}}

%----------------------------------------------------------------------------------------
%	INFORMATIONS
%----------------------------------------------------------------------------------------

\author{Alexandre Petit-Pas}
\title{Mémoire de fin d'études}

\begin{document}

%\subfile{cover/page_garde}
%----------------------------------------------------------------------------------------
%	TITLE PAGE
%----------------------------------------------------------------------------------------

\setlength{\parindent}{0cm}
\setlength{\parskip}{1ex plus 0.5ex minus 0.2ex}
\newcommand{\hsp}{\hspace{20pt}}
\newcommand{\HRule}{\rule{\linewidth}{0.5mm}}

\begin{titlepage}
  \begin{sffamily}
  \begin{center}

    \textsc{\LARGE MASTER MIAGE 2ème année \linebreak Université Paris Nanterre}\\[2cm]

    \textsc{\Large Mémoire de fin d'études}\\[1.5cm]

    \HRule \\[0.4cm]
    { \huge \bfseries Étude comparative d'algorithmes pour le classement de chiffres manuscrits\\[0.4cm] }

    \HRule \\[2cm]
    \vfill
    \begin{center}
    \includegraphics[scale=0.40]{img/logo_nanterre.jpg}
    \end{center}
    \hspace{2cm}
    
    \vfill
  \begin{minipage}{0.3\textwidth}
      \begin{flushleft} \large
        \emph{Auteur :}\\ \textsc{Alexandre PETIT-PAS}\\
      \end{flushleft}
    \end{minipage}
    \begin{minipage}{0.4\textwidth}
      \begin{flushright} \large
        \emph{Tuteur :}\\ \textsc{PR. Marie-Pierre GERVAIS}\\
      \end{flushright}
    \end{minipage}
    \vfill
  \end{center}
  \end{sffamily}
\end{titlepage}

\leavevmode\thispagestyle{empty}\newpage

%----------------------------------------------------------------------------------------
%	RESUME
%----------------------------------------------------------------------------------------

\begin{abstract}
	Le classement est une méthode connue de l'apprentissage supervisé qui permet par exemple de reconnaître des chiffres manuscrits. Les algorithmes CART, C4.5 et K-NN étudiés dans ce mémoire permettent de classer des numéros étudiants écrits à la main. Cependant, la diversité de ces algorithmes complexifie le choix. Dans ce mémoire, nous allons analyser ces algorithmes et les implémenter en utilisant MNIST comme base d'apprentissage. Les résultats nous montrent que CART est plus rapide mais K-NN est plus précis. Les résultats permettent de déterminer quel algorithme choisir dans ce cas de reconnaissance de chiffres mais ce choix n'est pas universel pour tous les modèles.
\end{abstract}

\leavevmode\thispagestyle{empty}\newpage

% \subfile{cover/remerciements}

%----------------------------------------------------------------------------------------
%	Remerciement
%----------------------------------------------------------------------------------------

\begin{acknowledgements}
Je tiens à remercier Mme Marie-Pierre Gervais et M. Emmanuel Hyon pour l'aide et les conseils précieux qu'ils m'ont apportés durant la rédaction du mémoire. 

Un grand merci à toute ma famille qui m'a soutenu jusqu'à la dernière ligne et qui m'a aidé à la relecture.

Enfin je tiens à remercier toutes les personnes qui m'ont aidé d'une quelconque manière à l'écriture de ce mémoire.
\end{acknowledgements}

\leavevmode\thispagestyle{empty}\newpage

%----------------------------------------------------------------------------------------
%	SOMMAIRE
%----------------------------------------------------------------------------------------

\shorttoc{Sommaire}{1}

%----------------------------------------------------------------------------------------
%	INTRODUCTION
%----------------------------------------------------------------------------------------

\subfile{introduction}

%----------------------------------------------------------------------------------------
%	CHAPITRES
%----------------------------------------------------------------------------------------
\subfile{chapitres/chapitre1}

\subfile{chapitres/chapitre3}

\subfile{chapitres/chapitre4}


%----------------------------------------------------------------------------------------
%	CONCLUSION
%----------------------------------------------------------------------------------------
\subfile{conclusion}

%----------------------------------------------------------------------------------------
%	ANNEXES
%----------------------------------------------------------------------------------------

\subfile{annexes/annexes}

%----------------------------------------------------------------------------------------
%	BIBLIOGRAPHIE
%----------------------------------------------------------------------------------------

\printbibliography[heading=bibintoc]
\newpage

%----------------------------------------------------------------------------------------
%	TABLE DES MATIERES
%----------------------------------------------------------------------------------------

\tableofcontents

\listoffigures
\listoftables

\end{document}